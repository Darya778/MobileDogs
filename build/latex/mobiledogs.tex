%% Generated by Sphinx.
\def\sphinxdocclass{report}
\documentclass[letterpaper,10pt,russian]{sphinxmanual}
\ifdefined\pdfpxdimen
   \let\sphinxpxdimen\pdfpxdimen\else\newdimen\sphinxpxdimen
\fi \sphinxpxdimen=.75bp\relax
\ifdefined\pdfimageresolution
    \pdfimageresolution= \numexpr \dimexpr1in\relax/\sphinxpxdimen\relax
\fi
%% let collapsible pdf bookmarks panel have high depth per default
\PassOptionsToPackage{bookmarksdepth=5}{hyperref}

\PassOptionsToPackage{booktabs}{sphinx}
\PassOptionsToPackage{colorrows}{sphinx}

\PassOptionsToPackage{warn}{textcomp}
\usepackage[utf8]{inputenc}
\ifdefined\DeclareUnicodeCharacter
% support both utf8 and utf8x syntaxes
  \ifdefined\DeclareUnicodeCharacterAsOptional
    \def\sphinxDUC#1{\DeclareUnicodeCharacter{"#1}}
  \else
    \let\sphinxDUC\DeclareUnicodeCharacter
  \fi
  \sphinxDUC{00A0}{\nobreakspace}
  \sphinxDUC{2500}{\sphinxunichar{2500}}
  \sphinxDUC{2502}{\sphinxunichar{2502}}
  \sphinxDUC{2514}{\sphinxunichar{2514}}
  \sphinxDUC{251C}{\sphinxunichar{251C}}
  \sphinxDUC{2572}{\textbackslash}
\fi
\usepackage{cmap}
\usepackage[T1]{fontenc}
\usepackage{amsmath,amssymb,amstext}
\usepackage{babel}





\usepackage[Sonny]{fncychap}
\ChNameVar{\Large\normalfont\sffamily}
\ChTitleVar{\Large\normalfont\sffamily}
\usepackage{sphinx}

\fvset{fontsize=auto}
\usepackage{geometry}


% Include hyperref last.
\usepackage{hyperref}
% Fix anchor placement for figures with captions.
\usepackage{hypcap}% it must be loaded after hyperref.
% Set up styles of URL: it should be placed after hyperref.
\urlstyle{same}

\addto\captionsrussian{\renewcommand{\contentsname}{Contents:}}

\usepackage{sphinxmessages}
\setcounter{tocdepth}{1}



\title{MobileDogs}
\date{июн. 01, 2024}
\release{1.0}
\author{Kuzmin Ilia, Nikiforova Darya}
\newcommand{\sphinxlogo}{\vbox{}}
\renewcommand{\releasename}{Выпуск}
\makeindex
\begin{document}

\ifdefined\shorthandoff
  \ifnum\catcode`\=\string=\active\shorthandoff{=}\fi
  \ifnum\catcode`\"=\active\shorthandoff{"}\fi
\fi

\pagestyle{empty}
\sphinxmaketitle
\pagestyle{plain}
\sphinxtableofcontents
\pagestyle{normal}
\phantomsection\label{\detokenize{index::doc}}
\sphinxAtStartPar
Документация проекта MobileDogs:
\begin{enumerate}
\sphinxsetlistlabels{\arabic}{enumi}{enumii}{}{.}%
\item {} 
\sphinxAtStartPar
Описание базы данных

\end{enumerate}

\sphinxAtStartPar
src/database.md
\begin{enumerate}
\sphinxsetlistlabels{\arabic}{enumi}{enumii}{}{.}%
\setcounter{enumi}{1}
\item {} 
\sphinxAtStartPar
Описание функций для пользователей

\end{enumerate}

\sphinxAtStartPar
src/users/users.md
\begin{enumerate}
\sphinxsetlistlabels{\arabic}{enumi}{enumii}{}{.}%
\setcounter{enumi}{2}
\item {} 
\sphinxAtStartPar
Описание функций для ошейников

\end{enumerate}

\sphinxAtStartPar
src/devices/devices.md
\begin{enumerate}
\sphinxsetlistlabels{\arabic}{enumi}{enumii}{}{.}%
\setcounter{enumi}{3}
\item {} 
\sphinxAtStartPar
Описание функций для заданий

\end{enumerate}

\sphinxAtStartPar
src/tasks/tasks.md
\begin{description}
\sphinxlineitem{. automodule:: mobiledogs}\begin{quote}\begin{description}
\sphinxlineitem{members}
\end{description}\end{quote}

\end{description}

\sphinxstepscope




\chapter{Идея проекта:}
\label{\detokenize{README:id1}}
\sphinxAtStartPar
Проект «Mobile Dogs» представляет собой систему для отслеживания бездомных собак с помощью носимых устройств (ошейников).
Каждый пользователь может выбрать себе некоторое количество бездомных собак для отслеживания.
Персонал устанавливает новые ошейники на бездомных собак.
Пользователи могут наблюдать за собаками, давать задания другим пользователям, а также получать данные о собаках через приложение.
Персонал устанавливает новые ошейники на бездомных собак.
Есть базовые станции LoRa для связи носимых устройств (ошейников).
Есть регистрация пользователей и ошейников. Носимые устройства регистрирует только персонал.


\chapter{Сценарии:}
\label{\detokenize{README:id2}}\begin{enumerate}
\sphinxsetlistlabels{\arabic}{enumi}{enumii}{}{.}%
\item {} 
\sphinxAtStartPar
Регистрация пользователей:
\begin{itemize}
\item {} 
\sphinxAtStartPar
Пользователи могут зарегистрироваться, указав свои данные и получив уникальный идентификатор.

\end{itemize}

\item {} 
\sphinxAtStartPar
Регистрация ошейников:
\begin{itemize}
\item {} 
\sphinxAtStartPar
Персонал может зарегистрировать новые ошейники, присваивая им уникальные идентификаторы и привязывая к конкретным собакам.

\end{itemize}

\item {} 
\sphinxAtStartPar
Оповещение пользователей:
\begin{itemize}
\item {} 
\sphinxAtStartPar
Пользователи получают данные о местоположении и состоянии собак через приложение.

\end{itemize}

\item {} 
\sphinxAtStartPar
Модерирование:
\begin{itemize}
\item {} 
\sphinxAtStartPar
Персонал следит за состоянием ошейников, информацией о пользователях и собаках, и при необходимости, модерирует контент.

\end{itemize}

\item {} 
\sphinxAtStartPar
Задание от одних пользователям другим:
\begin{itemize}
\item {} 
\sphinxAtStartPar
Пользователи могут давать задания другим пользователям по уходу за определенными собаками, кормлением или выполнению других действий.

\end{itemize}

\item {} 
\sphinxAtStartPar
Привязка ошейников и пользователей:
\begin{itemize}
\item {} 
\sphinxAtStartPar
Пользователи могут «привязывать» зарегистрированные ошейники к своему аккаунту для отслеживания конкретных собак.

\end{itemize}

\end{enumerate}


\chapter{Необходимые запросы для реализации сценариев:}
\label{\detokenize{README:id3}}\begin{enumerate}
\sphinxsetlistlabels{\arabic}{enumi}{enumii}{}{.}%
\item {} 
\sphinxAtStartPar
POST request для регистрации пользователей

\item {} 
\sphinxAtStartPar
PUT request для изменения пользователя

\item {} 
\sphinxAtStartPar
DELETE request для удаления пользователя

\item {} 
\sphinxAtStartPar
POST request для добавления задания

\item {} 
\sphinxAtStartPar
PUT request для изменения задания

\item {} 
\sphinxAtStartPar
DELETE request для удаления задания

\item {} 
\sphinxAtStartPar
POST request для добавления ошейника

\item {} 
\sphinxAtStartPar
PUT request для изменения ошейника

\item {} 
\sphinxAtStartPar
DELETE request для удаления ошейника

\item {} 
\sphinxAtStartPar
GET request для получения списка всех ошейников

\item {} 
\sphinxAtStartPar
POST request для добавления координат

\item {} 
\sphinxAtStartPar
PUT request для изменения координат

\item {} 
\sphinxAtStartPar
DELETE request для удаления координат

\item {} 
\sphinxAtStartPar
GET request для возвращения всех координат ошейника по заданному времени

\item {} 
\sphinxAtStartPar
POST request для добавления связи пользователя с ошейником

\item {} 
\sphinxAtStartPar
PUT request для изменения связи пользователя с ошейником

\item {} 
\sphinxAtStartPar
DELETE request для удаления связи пользователя с ошейником

\end{enumerate}


\chapter{Инструкции для установки зависимостей и запуска приложения:}
\label{\detokenize{README:id4}}

\section{Установка зависимостей}
\label{\detokenize{README:id5}}\begin{enumerate}
\sphinxsetlistlabels{\arabic}{enumi}{enumii}{}{.}%
\item {} 
\sphinxAtStartPar
Убедитесь, что у вас установлены Python версии 3.6 или выше, а также pip версии 10.0 или выше.

\item {} 
\sphinxAtStartPar
Создайте виртуальное окружение, чтобы изолировать зависимости приложения от других проектов. Используйте команду:

\end{enumerate}

\begin{sphinxVerbatim}[commandchars=\\\{\}]
\PYG{n}{python3} \PYG{o}{\PYGZhy{}}\PYG{n}{m} \PYG{n}{venv} \PYG{n}{venv}
\end{sphinxVerbatim}
\begin{enumerate}
\sphinxsetlistlabels{\arabic}{enumi}{enumii}{}{.}%
\setcounter{enumi}{2}
\item {} 
\sphinxAtStartPar
Активируйте виртуальное окружение:

\end{enumerate}

\begin{sphinxVerbatim}[commandchars=\\\{\}]
\PYG{n}{source} \PYG{n}{venv}\PYG{o}{/}\PYG{n+nb}{bin}\PYG{o}{/}\PYG{n}{activate}
\end{sphinxVerbatim}
\begin{enumerate}
\sphinxsetlistlabels{\arabic}{enumi}{enumii}{}{.}%
\setcounter{enumi}{3}
\item {} 
\sphinxAtStartPar
Установите зависимости из файла requirements.txt. Запустите команду:

\end{enumerate}

\begin{sphinxVerbatim}[commandchars=\\\{\}]
\PYG{n}{pip} \PYG{n}{install} \PYG{o}{\PYGZhy{}}\PYG{n}{r} \PYG{n}{requirements}\PYG{o}{.}\PYG{n}{txt} 
\end{sphinxVerbatim}


\section{Доп. установки и настройки}
\label{\detokenize{README:id6}}\begin{enumerate}
\sphinxsetlistlabels{\arabic}{enumi}{enumii}{}{.}%
\item {} 
\sphinxAtStartPar
Настройте ELK

\item {} 
\sphinxAtStartPar
Настройте Filebeat

\end{enumerate}


\section{Запуск приложения}
\label{\detokenize{README:id7}}\begin{enumerate}
\sphinxsetlistlabels{\arabic}{enumi}{enumii}{}{.}%
\item {} 
\sphinxAtStartPar
Клонируйте репозиторий:

\end{enumerate}

\begin{sphinxVerbatim}[commandchars=\\\{\}]
\PYG{n}{sudo} \PYG{n}{apt} \PYG{n}{update}
\PYG{n}{sudo} \PYG{n}{apt} \PYG{n}{install} \PYG{n}{git}
\PYG{n}{git} \PYG{n}{clone} \PYG{n}{URL}
\end{sphinxVerbatim}
\begin{enumerate}
\sphinxsetlistlabels{\arabic}{enumi}{enumii}{}{.}%
\setcounter{enumi}{1}
\item {} 
\sphinxAtStartPar
Перейдите в каталог приложения (\textasciitilde{}/MobileDogs/src).

\item {} 
\sphinxAtStartPar
Создайте новую виртуальную среду и активируйте:

\end{enumerate}

\begin{sphinxVerbatim}[commandchars=\\\{\}]
\PYG{n}{python} \PYG{o}{\PYGZhy{}}\PYG{n}{m} \PYG{n}{venv} \PYG{n}{venv}
\PYG{n}{source} \PYG{o}{.}\PYG{n}{venv}\PYG{o}{/}\PYG{n+nb}{bin}\PYG{o}{/}\PYG{n}{activate}
\end{sphinxVerbatim}
\begin{enumerate}
\sphinxsetlistlabels{\arabic}{enumi}{enumii}{}{.}%
\setcounter{enumi}{3}
\item {} 
\sphinxAtStartPar
Запустите приложение с помощью команды:

\end{enumerate}

\begin{sphinxVerbatim}[commandchars=\\\{\}]
\PYG{n}{uvicorn} \PYG{n}{main}\PYG{p}{:}\PYG{n}{app} \PYG{o}{\PYGZhy{}}\PYG{o}{\PYGZhy{}}\PYG{n}{reload}
\end{sphinxVerbatim}
\begin{enumerate}
\sphinxsetlistlabels{\arabic}{enumi}{enumii}{}{.}%
\setcounter{enumi}{4}
\item {} 
\sphinxAtStartPar
Запустите Filebeat в отдельном окне и проверить статус работы всех серверов (в файле \sphinxcode{\sphinxupquote{/etc/filebeat/filebeat.yml}} в \sphinxcode{\sphinxupquote{filebeat.inputs}} в \sphinxcode{\sphinxupquote{parth}} \textasciitilde{}/MobileDogs/src/log/*.json)

\end{enumerate}

\begin{sphinxVerbatim}[commandchars=\\\{\}]
\PYG{n}{sudo} \PYG{n}{systemctl} \PYG{n}{status} \PYG{n}{elasticsearch}\PYG{o}{.}\PYG{n}{service}
\PYG{n}{sudo} \PYG{n}{systemctl} \PYG{n}{status} \PYG{n}{kibana}\PYG{o}{.}\PYG{n}{service}
\PYG{n}{sudo} \PYG{n}{systemctl} \PYG{n}{status} \PYG{n}{logstash}\PYG{o}{.}\PYG{n}{service}
\PYG{n}{sudo} \PYG{n}{filebeat} \PYG{o}{\PYGZhy{}}\PYG{n}{e}
\end{sphinxVerbatim}
\begin{enumerate}
\sphinxsetlistlabels{\arabic}{enumi}{enumii}{}{.}%
\setcounter{enumi}{5}
\item {} 
\sphinxAtStartPar
Чтобы выйти из приложения, нажмите \sphinxcode{\sphinxupquote{Ctrl + C}} в терминале, в котором оно запущено. Чтобы деактивировать виртуальное окружение, запустите команду: \sphinxcode{\sphinxupquote{deactivate}}

\end{enumerate}

\sphinxstepscope

\sphinxAtStartPar
psycopg2\sphinxhyphen{}binary\textless{}=2.8.3
ecs\_logging\textless{}=2.1.0
fastapi\textless{}=0.103.0
Flask\textless{}=1.1.1
markdown\sphinxhyphen{}it\sphinxhyphen{}py\textless{}=2.2.0
matplotlib\textless{}=3.1.3
myst\sphinxhyphen{}parser\textless{}=1.0.0
networkx\textless{}=2.4
passlib\textless{}=1.7.4
python\sphinxhyphen{}dateutil==2.8.1
python\sphinxhyphen{}jsonrpc\sphinxhyphen{}server\textless{}=0.3.4
python\sphinxhyphen{}language\sphinxhyphen{}server\textless{}=0.31.7
python\sphinxhyphen{}logstash\sphinxhyphen{}async\textless{}=3.0.0
pydantic\textless{}=2.5.3
Requests\textless{}=2.31.0
Sphinx\textless{}=5.3.0
sphinxcontrib\sphinxhyphen{}applehelp\textless{}=1.0.2
sphinxcontrib\sphinxhyphen{}devhelp\textless{}=1.0.2
sphinxcontrib\sphinxhyphen{}htmlhelp\textless{}=2.0.5
sphinxcontrib\sphinxhyphen{}jsmath\textless{}=1.0.1
sphinxcontrib\sphinxhyphen{}qthelp\textless{}=1.0.7
sphinxcontrib\sphinxhyphen{}serializinghtml\textless{}=1.1.10
SQLAlchemy\textless{}=2.0.30
sqlalchemy\sphinxhyphen{}orm\textless{}=1.2.10
uvicorn\textless{}=0.30.0
pathlib
h5py\textgreater{}=2.9.0
clyent==1.2.1
nbformat==5.4.0
requests==2.28.1
jedi\textgreater{}=0.16

\sphinxstepscope

\sphinxAtStartPar
LICENSE
README.md
setup.py
MobileDogs.egg\sphinxhyphen{}info/PKG\sphinxhyphen{}INFO
MobileDogs.egg\sphinxhyphen{}info/SOURCES.txt
MobileDogs.egg\sphinxhyphen{}info/dependency\_links.txt
MobileDogs.egg\sphinxhyphen{}info/requires.txt
MobileDogs.egg\sphinxhyphen{}info/top\_level.txt
src/\_\_init\_\_.py
src/conf.py
src/database.py
src/database\_manager.py
src/main.py

\sphinxstepscope

\sphinxstepscope

\sphinxAtStartPar
psycopg2\sphinxhyphen{}binary\textless{}=2.8.3
ecs\_logging\textless{}=2.1.0
fastapi\textless{}=0.103.0
Flask\textless{}=1.1.1
markdown\sphinxhyphen{}it\sphinxhyphen{}py\textless{}=2.2.0
matplotlib\textless{}=3.1.3
myst\sphinxhyphen{}parser\textless{}=1.0.0
networkx\textless{}=2.4
passlib\textless{}=1.7.4
python\sphinxhyphen{}dateutil==2.8.1
python\sphinxhyphen{}jsonrpc\sphinxhyphen{}server\textless{}=0.3.4
python\sphinxhyphen{}language\sphinxhyphen{}server\textless{}=0.31.7
python\sphinxhyphen{}logstash\sphinxhyphen{}async\textless{}=3.0.0
pydantic\textless{}=2.5.3
Requests\textless{}=2.31.0
Sphinx\textless{}=5.3.0
sphinxcontrib\sphinxhyphen{}applehelp\textless{}=1.0.2
sphinxcontrib\sphinxhyphen{}devhelp\textless{}=1.0.2
sphinxcontrib\sphinxhyphen{}htmlhelp\textless{}=2.0.5
sphinxcontrib\sphinxhyphen{}jsmath\textless{}=1.0.1
sphinxcontrib\sphinxhyphen{}qthelp\textless{}=1.0.7
sphinxcontrib\sphinxhyphen{}serializinghtml\textless{}=1.1.10
SQLAlchemy\textless{}=2.0.30
sqlalchemy\sphinxhyphen{}orm\textless{}=1.2.10
uvicorn\textless{}=0.30.0
pathlib
h5py\textgreater{}=2.9.0
clyent==1.2.1
nbformat==5.4.0
requests==2.28.1
jedi\textgreater{}=0.16

\sphinxAtStartPar
{[}test{]}
pytest
coverage

\sphinxstepscope

\sphinxAtStartPar
src

\sphinxstepscope


\chapter{Описание базы данных}
\label{\detokenize{src/database:id1}}\label{\detokenize{src/database::doc}}
\sphinxAtStartPar
В базе данных есть 5 таблиц:

\sphinxAtStartPar
\sphinxstylestrong{1. \sphinxcode{\sphinxupquote{users}}}
\begin{itemize}
\item {} 
\sphinxAtStartPar
\sphinxstylestrong{id:}  Идентификатор пользователя (первичный ключ).

\item {} 
\sphinxAtStartPar
\sphinxstylestrong{login:}  Логин пользователя (уникальный).

\item {} 
\sphinxAtStartPar
\sphinxstylestrong{password:}  Хэшированный пароль пользователя.

\item {} 
\sphinxAtStartPar
\sphinxstylestrong{email:}  Адрес электронной почты пользователя (уникальный).

\item {} 
\sphinxAtStartPar
\sphinxstylestrong{gender:}  Пол пользователя (необязательно).

\item {} 
\sphinxAtStartPar
\sphinxstylestrong{phone:}  Номер телефона пользователя (необязательно).

\item {} 
\sphinxAtStartPar
\sphinxstylestrong{birthday:}  Дата рождения пользователя (необязательно).

\item {} 
\sphinxAtStartPar
\sphinxstylestrong{registration\_date:}  Дата регистрации пользователя.

\item {} 
\sphinxAtStartPar
\sphinxstylestrong{deletion\_date:}  Дата удаления (архивирования) пользователя (необязательно).

\item {} 
\sphinxAtStartPar
\sphinxstylestrong{archived:}  Флаг, указывающий, был ли пользователь удален (архивирован) (по умолчанию False).

\end{itemize}

\sphinxAtStartPar
\sphinxstylestrong{2. \sphinxcode{\sphinxupquote{tasks}}}
\begin{itemize}
\item {} 
\sphinxAtStartPar
\sphinxstylestrong{id:} Идентификатор задачи (первичный ключ).

\item {} 
\sphinxAtStartPar
\sphinxstylestrong{id\_user\_1:}  Идентификатор первого пользователя, связанного с задачей (используется внешний ключ для связи с таблицей \sphinxcode{\sphinxupquote{users}}).

\item {} 
\sphinxAtStartPar
\sphinxstylestrong{id\_user\_2:}  Идентификатор второго пользователя, связанного с задачей (используется внешний ключ для связи с таблицей \sphinxcode{\sphinxupquote{users}}).

\item {} 
\sphinxAtStartPar
\sphinxstylestrong{id\_task:}  Дополнительный идентификатор задачи (необязательно).

\item {} 
\sphinxAtStartPar
\sphinxstylestrong{confirm:}  Флаг, указывающий, подтверждена ли задача (по умолчанию False).

\item {} 
\sphinxAtStartPar
\sphinxstylestrong{send\_date:}  Дата отправки задачи.

\item {} 
\sphinxAtStartPar
\sphinxstylestrong{completion\_date:}  Дата завершения задачи (необязательно).

\item {} 
\sphinxAtStartPar
\sphinxstylestrong{task\_text:} Текст задачи.

\item {} 
\sphinxAtStartPar
\sphinxstylestrong{archived:}  Флаг, указывающий, была ли задача удалена (архивирована) (по умолчанию False).

\end{itemize}

\sphinxAtStartPar
\sphinxstylestrong{3. \sphinxcode{\sphinxupquote{dogs\_collar}}}
\begin{itemize}
\item {} 
\sphinxAtStartPar
\sphinxstylestrong{id:}  Идентификатор ошейника (первичный ключ).

\item {} 
\sphinxAtStartPar
\sphinxstylestrong{uni\_num\_dog:}  Уникальный номер ошейника.

\item {} 
\sphinxAtStartPar
\sphinxstylestrong{name\_dog:}  Кличка собаки.

\item {} 
\sphinxAtStartPar
\sphinxstylestrong{feeling\_hungry:}  Флаг, указывающий, голодна ли собака (необязательно).

\item {} 
\sphinxAtStartPar
\sphinxstylestrong{health\_status:}  Состояние здоровья собаки (необязательно).

\item {} 
\sphinxAtStartPar
\sphinxstylestrong{registration\_date:}  Дата регистрации ошейника.

\item {} 
\sphinxAtStartPar
\sphinxstylestrong{deletion\_date:}  Дата удаления (архивирования) ошейника (необязательно).

\item {} 
\sphinxAtStartPar
\sphinxstylestrong{archived:}  Флаг, указывающий, был ли ошейник удален (архивирован) (по умолчанию False).

\end{itemize}

\sphinxAtStartPar
\sphinxstylestrong{4. \sphinxcode{\sphinxupquote{users\_dogs\_collar}}}
\begin{itemize}
\item {} 
\sphinxAtStartPar
\sphinxstylestrong{id:}  Идентификатор связи между пользователем и ошейником (первичный ключ).

\item {} 
\sphinxAtStartPar
\sphinxstylestrong{id\_user:}  Идентификатор пользователя (используется внешний ключ для связи с таблицей \sphinxcode{\sphinxupquote{users}}).

\item {} 
\sphinxAtStartPar
\sphinxstylestrong{id\_collar:}  Идентификатор ошейника (используется внешний ключ для связи с таблицей \sphinxcode{\sphinxupquote{dogs\_collar}}).

\item {} 
\sphinxAtStartPar
\sphinxstylestrong{binding\_date:}  Дата связи между пользователем и ошейником.

\item {} 
\sphinxAtStartPar
\sphinxstylestrong{unbinding\_date:}  Дата разрыва связи между пользователем и ошейником (необязательно).

\item {} 
\sphinxAtStartPar
\sphinxstylestrong{archived:}  Флаг, указывающий, была ли связь удалена (архивирована) (по умолчанию False).

\end{itemize}

\sphinxAtStartPar
\sphinxstylestrong{5. \sphinxcode{\sphinxupquote{coordinates}}}
\begin{itemize}
\item {} 
\sphinxAtStartPar
\sphinxstylestrong{id:}  Идентификатор координат (первичный ключ).

\item {} 
\sphinxAtStartPar
\sphinxstylestrong{collar\_id:}  Идентификатор ошейника (используется внешний ключ для связи с таблицей \sphinxcode{\sphinxupquote{dogs\_collar}}).

\item {} 
\sphinxAtStartPar
\sphinxstylestrong{latitude:}  Широта.

\item {} 
\sphinxAtStartPar
\sphinxstylestrong{longitude:}  Долгота.

\item {} 
\sphinxAtStartPar
\sphinxstylestrong{timestamp:}  Время фиксации координат.

\end{itemize}

\sphinxAtStartPar
\sphinxstylestrong{Схемы отношений:}
\begin{itemize}
\item {} 
\sphinxAtStartPar
\sphinxstylestrong{\sphinxcode{\sphinxupquote{users}}} имеет отношение \sphinxstylestrong{«один ко многим»}  с \sphinxstylestrong{\sphinxcode{\sphinxupquote{tasks}}} (через \sphinxcode{\sphinxupquote{id\_user\_1}} и \sphinxcode{\sphinxupquote{id\_user\_2}}).

\item {} 
\sphinxAtStartPar
\sphinxstylestrong{\sphinxcode{\sphinxupquote{users}}} имеет отношение \sphinxstylestrong{«один ко многим»}  с \sphinxstylestrong{\sphinxcode{\sphinxupquote{users\_dogs\_collar}}} (через \sphinxcode{\sphinxupquote{id\_user}}).

\item {} 
\sphinxAtStartPar
\sphinxstylestrong{\sphinxcode{\sphinxupquote{dogs\_collar}}} имеет отношение \sphinxstylestrong{«один ко многим»}  с \sphinxstylestrong{\sphinxcode{\sphinxupquote{users\_dogs\_collar}}} (через \sphinxcode{\sphinxupquote{id\_collar}}).

\item {} 
\sphinxAtStartPar
\sphinxstylestrong{\sphinxcode{\sphinxupquote{dogs\_collar}}} имеет отношение \sphinxstylestrong{«один ко многим»}  с \sphinxstylestrong{\sphinxcode{\sphinxupquote{coordinates}}} (через \sphinxcode{\sphinxupquote{collar\_id}}).

\end{itemize}


\section{\sphinxstylestrong{Функциональность:}}
\label{\detokenize{src/database:id2}}
\sphinxAtStartPar
База данных может быть использована для:
\begin{itemize}
\item {} 
\sphinxAtStartPar
\sphinxstylestrong{Управления пользователями:}  Регистрации, входа, изменения профиля, архивирования.

\item {} 
\sphinxAtStartPar
\sphinxstylestrong{Планирования задач:}  Создание, подтверждение, отправка, завершение, архивирование.

\item {} 
\sphinxAtStartPar
\sphinxstylestrong{Отслеживания собак:}  Регистрация ошейников, запись координат, управление связью между ошейником и пользователем.

\end{itemize}


\section{\sphinxstylestrong{Дополнительные замечания:}}
\label{\detokenize{src/database:id3}}\begin{itemize}
\item {} 
\sphinxAtStartPar
\sphinxcode{\sphinxupquote{archived}}\sphinxhyphen{}флаги позволяют архивировать данные, а не полностью удалять их. Это может быть полезно для восстановления данных или для аналитики.

\item {} 
\sphinxAtStartPar
\sphinxcode{\sphinxupquote{TIMESTAMP}}  с \sphinxcode{\sphinxupquote{default=\textquotesingle{}now()\textquotesingle{}}}  используется для автоматического заполнения поля датой и временем при добавлении новой записи.

\item {} 
\sphinxAtStartPar
Внешние ключи обеспечивают целостность данных, гарантируя, что записи связаны между собой правильно.

\end{itemize}

\sphinxstepscope


\chapter{Функции для работы с ошейниками}
\label{\detokenize{src/devices/devices:id1}}\label{\detokenize{src/devices/devices::doc}}
\sphinxAtStartPar
\sphinxstylestrong{1. \sphinxcode{\sphinxupquote{create\_dog\_collar(collar: DogCollarCreate, db: Session = Depends(get\_db))}}}

\sphinxAtStartPar
Описание: Эта функция отвечает за создание новых ошейников.

\sphinxAtStartPar
Параметры:
\begin{itemize}
\item {} 
\sphinxAtStartPar
\sphinxcode{\sphinxupquote{collar: DogCollarCreate}}: Данные нового ошейника, которые передаются в виде объекта \sphinxcode{\sphinxupquote{DogCollarCreate}}.

\item {} 
\sphinxAtStartPar
\sphinxcode{\sphinxupquote{db: Session}}: Сессия базы данных, полученная из контекста зависимости \sphinxcode{\sphinxupquote{get\_db}}.

\end{itemize}

\sphinxAtStartPar
Действия:
\begin{enumerate}
\sphinxsetlistlabels{\arabic}{enumi}{enumii}{}{.}%
\item {} 
\sphinxAtStartPar
Вызывает функцию \sphinxcode{\sphinxupquote{db\_manager.add\_dog\_collar}} для добавления нового ошейника в базу данных.

\item {} 
\sphinxAtStartPar
Если функция \sphinxcode{\sphinxupquote{db\_manager.add\_dog\_collar}} успешно добавляет ошейник, функция записывает информацию об успешном добавлении в лог и возвращает ID ошейника и сообщение об успешном добавлении.

\item {} 
\sphinxAtStartPar
Обрабатывает любые исключения, возникшие в процессе добавления, записывает их в лог и генерирует исключение \sphinxcode{\sphinxupquote{HTTPException}} с кодом 500 и описанием ошибки.

\end{enumerate}

\sphinxAtStartPar
\sphinxstylestrong{2. \sphinxcode{\sphinxupquote{update\_dog\_collar(collar\_id: int, collar\_update: DogCollarUpdate, db: Session = Depends(get\_db))}}}

\sphinxAtStartPar
Описание: Эта функция отвечает за обновление данных существующего ошейника.

\sphinxAtStartPar
Параметры:
\begin{itemize}
\item {} 
\sphinxAtStartPar
\sphinxcode{\sphinxupquote{collar\_id: int}}: ID ошейника, который нужно обновить.

\item {} 
\sphinxAtStartPar
\sphinxcode{\sphinxupquote{collar\_update: DogCollarUpdate}}: Данные для обновления, которые передаются в виде объекта \sphinxcode{\sphinxupquote{DogCollarUpdate}}.

\item {} 
\sphinxAtStartPar
\sphinxcode{\sphinxupquote{db: Session}}: Сессия базы данных, полученная из контекста зависимости \sphinxcode{\sphinxupquote{get\_db}}.

\end{itemize}

\sphinxAtStartPar
Действия:
\begin{enumerate}
\sphinxsetlistlabels{\arabic}{enumi}{enumii}{}{.}%
\item {} 
\sphinxAtStartPar
Вызывает функцию \sphinxcode{\sphinxupquote{db\_manager.update\_dog\_collar}} для обновления данных ошейника в базе данных.

\item {} 
\sphinxAtStartPar
Если функция \sphinxcode{\sphinxupquote{db\_manager.update\_dog\_collar}} успешно обновляет данные, функция записывает информацию об успешном обновлении в лог и возвращает сообщение об успешном обновлении.

\item {} 
\sphinxAtStartPar
Обрабатывает любые исключения, возникшие в процессе обновления, записывает их в лог и генерирует исключение \sphinxcode{\sphinxupquote{HTTPException}} с кодом 500 и описанием ошибки.

\end{enumerate}

\sphinxAtStartPar
\sphinxstylestrong{3. \sphinxcode{\sphinxupquote{delete\_dog\_collar(collar\_id: int, db: Session = Depends(get\_db))}}}

\sphinxAtStartPar
Описание: Эта функция отвечает за удаление (архивацию) ошейника.

\sphinxAtStartPar
Параметры:
\begin{itemize}
\item {} 
\sphinxAtStartPar
\sphinxcode{\sphinxupquote{collar\_id: int}}: ID ошейника, который нужно удалить.

\item {} 
\sphinxAtStartPar
\sphinxcode{\sphinxupquote{db: Session}}: Сессия базы данных, полученная из контекста зависимости \sphinxcode{\sphinxupquote{get\_db}}.

\end{itemize}

\sphinxAtStartPar
Действия:
\begin{enumerate}
\sphinxsetlistlabels{\arabic}{enumi}{enumii}{}{.}%
\item {} 
\sphinxAtStartPar
Вызывает функцию \sphinxcode{\sphinxupquote{db\_manager.archive\_dog\_collar}} для архивирования ошейника в базе данных.

\item {} 
\sphinxAtStartPar
Если функция \sphinxcode{\sphinxupquote{db\_manager.archive\_dog\_collar}} успешно архивирует ошейник, функция записывает информацию об успешном удалении в лог и возвращает сообщение об успешном удалении.

\item {} 
\sphinxAtStartPar
Обрабатывает любые исключения, возникшие в процессе удаления, записывает их в лог и генерирует исключение \sphinxcode{\sphinxupquote{HTTPException}} с кодом 500 и описанием ошибки.

\end{enumerate}

\sphinxAtStartPar
\sphinxstylestrong{4. \sphinxcode{\sphinxupquote{get\_all\_dog\_collars(db: Session = Depends(get\_db))}}}

\sphinxAtStartPar
Описание: Эта функция возвращает список всех ошейников.

\sphinxAtStartPar
Параметры:
\begin{itemize}
\item {} 
\sphinxAtStartPar
\sphinxcode{\sphinxupquote{db: Session}}: Сессия базы данных, полученная из контекста зависимости \sphinxcode{\sphinxupquote{get\_db}}.

\end{itemize}

\sphinxAtStartPar
Действия:
\begin{enumerate}
\sphinxsetlistlabels{\arabic}{enumi}{enumii}{}{.}%
\item {} 
\sphinxAtStartPar
Вызывает функцию \sphinxcode{\sphinxupquote{db\_manager.query\_all\_dog\_collars}} для получения всех ошейников из базы данных.

\item {} 
\sphinxAtStartPar
Если функция \sphinxcode{\sphinxupquote{db\_manager.query\_all\_dog\_collars}} успешно возвращает список ошейников, функция записывает информацию об успешном получении в лог и возвращает список ошейников.

\item {} 
\sphinxAtStartPar
Обрабатывает любые исключения, возникшие в процессе получения, записывает их в лог и генерирует исключение \sphinxcode{\sphinxupquote{HTTPException}} с кодом 500 и описанием ошибки.

\end{enumerate}

\sphinxAtStartPar
\sphinxstylestrong{5. \sphinxcode{\sphinxupquote{create\_coordinate(coordinate: CoordinateCreate, db: Session = Depends(get\_db))}}}

\sphinxAtStartPar
Описание: Эта функция отвечает за создание новых координат.

\sphinxAtStartPar
Параметры:
\begin{itemize}
\item {} 
\sphinxAtStartPar
\sphinxcode{\sphinxupquote{coordinate: CoordinateCreate}}: Данные новых координат, которые передаются в виде объекта \sphinxcode{\sphinxupquote{CoordinateCreate}}.

\item {} 
\sphinxAtStartPar
\sphinxcode{\sphinxupquote{db: Session}}: Сессия базы данных, полученная из контекста зависимости \sphinxcode{\sphinxupquote{get\_db}}.

\end{itemize}

\sphinxAtStartPar
Действия:
\begin{enumerate}
\sphinxsetlistlabels{\arabic}{enumi}{enumii}{}{.}%
\item {} 
\sphinxAtStartPar
Вызывает функцию \sphinxcode{\sphinxupquote{db\_manager.add\_coordinate}} для добавления новых координат в базу данных.

\item {} 
\sphinxAtStartPar
Если функция \sphinxcode{\sphinxupquote{db\_manager.add\_coordinate}} успешно добавляет координаты, функция записывает информацию об успешном добавлении в лог и возвращает ID координат и сообщение об успешном добавлении.

\item {} 
\sphinxAtStartPar
Обрабатывает любые исключения, возникшие в процессе добавления, записывает их в лог и генерирует исключение \sphinxcode{\sphinxupquote{HTTPException}} с кодом 500 и описанием ошибки.

\end{enumerate}

\sphinxAtStartPar
\sphinxstylestrong{6. \sphinxcode{\sphinxupquote{update\_coordinate(coordinate\_id: int, coordinate\_update: CoordinateUpdate, db: Session = Depends(get\_db))}}}

\sphinxAtStartPar
Описание: Эта функция отвечает за обновление данных существующих координат.

\sphinxAtStartPar
Параметры:
\begin{itemize}
\item {} 
\sphinxAtStartPar
\sphinxcode{\sphinxupquote{coordinate\_id: int}}: ID координат, которые нужно обновить.

\item {} 
\sphinxAtStartPar
\sphinxcode{\sphinxupquote{coordinate\_update: CoordinateUpdate}}: Данные для обновления, которые передаются в виде объекта \sphinxcode{\sphinxupquote{CoordinateUpdate}}.

\item {} 
\sphinxAtStartPar
\sphinxcode{\sphinxupquote{db: Session}}: Сессия базы данных, полученная из контекста зависимости \sphinxcode{\sphinxupquote{get\_db}}.

\end{itemize}

\sphinxAtStartPar
Действия:
\begin{enumerate}
\sphinxsetlistlabels{\arabic}{enumi}{enumii}{}{.}%
\item {} 
\sphinxAtStartPar
Вызывает функцию \sphinxcode{\sphinxupquote{db\_manager.update\_coordinate}} для обновления данных координат в базе данных.

\item {} 
\sphinxAtStartPar
Если функция \sphinxcode{\sphinxupquote{db\_manager.update\_coordinate}} успешно обновляет данные, функция записывает информацию об успешном обновлении в лог и возвращает сообщение об успешном обновлении.

\item {} 
\sphinxAtStartPar
Обрабатывает любые исключения, возникшие в процессе обновления, записывает их в лог и генерирует исключение \sphinxcode{\sphinxupquote{HTTPException}} с кодом 500 и описанием ошибки.

\end{enumerate}

\sphinxAtStartPar
\sphinxstylestrong{7. \sphinxcode{\sphinxupquote{delete\_coordinate(coordinate\_id: int, db: Session = Depends(get\_db))}}}

\sphinxAtStartPar
Описание: Эта функция отвечает за удаление координат.

\sphinxAtStartPar
Параметры:
\begin{itemize}
\item {} 
\sphinxAtStartPar
\sphinxcode{\sphinxupquote{coordinate\_id: int}}: ID координат, которые нужно удалить.

\item {} 
\sphinxAtStartPar
\sphinxcode{\sphinxupquote{db: Session}}: Сессия базы данных, полученная из контекста зависимости \sphinxcode{\sphinxupquote{get\_db}}.

\end{itemize}

\sphinxAtStartPar
Действия:
\begin{enumerate}
\sphinxsetlistlabels{\arabic}{enumi}{enumii}{}{.}%
\item {} 
\sphinxAtStartPar
Вызывает функцию \sphinxcode{\sphinxupquote{db\_manager.delete\_coordinate}} для удаления координат из базы данных.

\item {} 
\sphinxAtStartPar
Если функция \sphinxcode{\sphinxupquote{db\_manager.delete\_coordinate}} успешно удаляет координаты, функция записывает информацию об успешном удалении в лог и возвращает сообщение об успешном удалении.

\item {} 
\sphinxAtStartPar
Обрабатывает любые исключения, возникшие в процессе удаления, записывает их в лог и генерирует исключение \sphinxcode{\sphinxupquote{HTTPException}} с кодом 500 и описанием ошибки.

\end{enumerate}

\sphinxAtStartPar
\sphinxstylestrong{8. \sphinxcode{\sphinxupquote{get\_track(collar\_id: int, start\_time: str, end\_time: str, db: Session = Depends(get\_db))}}}

\sphinxAtStartPar
Описание: Эта функция возвращает список координат для заданного ошейника за определенный период времени.

\sphinxAtStartPar
Параметры:
\begin{itemize}
\item {} 
\sphinxAtStartPar
\sphinxcode{\sphinxupquote{collar\_id: int}}: ID ошейника.

\item {} 
\sphinxAtStartPar
\sphinxcode{\sphinxupquote{start\_time: str}}: Начальное время периода.

\item {} 
\sphinxAtStartPar
\sphinxcode{\sphinxupquote{end\_time: str}}: Конечное время периода.

\item {} 
\sphinxAtStartPar
\sphinxcode{\sphinxupquote{db: Session}}: Сессия базы данных, полученная из контекста зависимости \sphinxcode{\sphinxupquote{get\_db}}.

\end{itemize}

\sphinxAtStartPar
Действия:
\begin{enumerate}
\sphinxsetlistlabels{\arabic}{enumi}{enumii}{}{.}%
\item {} 
\sphinxAtStartPar
Вызывает функцию \sphinxcode{\sphinxupquote{db\_manager.get\_track}} для получения списка координат из базы данных.

\item {} 
\sphinxAtStartPar
Если функция \sphinxcode{\sphinxupquote{db\_manager.get\_track}} успешно возвращает список координат, функция записывает информацию об успешном получении в лог и возвращает список координат.

\item {} 
\sphinxAtStartPar
Обрабатывает любые исключения, возникшие в процессе получения, записывает их в лог и генерирует исключение \sphinxcode{\sphinxupquote{HTTPException}} с кодом 500 и описанием ошибки.

\end{enumerate}

\sphinxAtStartPar
\sphinxstylestrong{9. \sphinxcode{\sphinxupquote{create\_user\_dog\_collar(binding: UserDogCollarCreate, db: Session = Depends(get\_db))}}}

\sphinxAtStartPar
Описание: Эта функция отвечает за создание новой связи между пользователем и ошейником.

\sphinxAtStartPar
Параметры:
\begin{itemize}
\item {} 
\sphinxAtStartPar
\sphinxcode{\sphinxupquote{binding: UserDogCollarCreate}}: Данные новой связи, которые передаются в виде объекта \sphinxcode{\sphinxupquote{UserDogCollarCreate}}.

\item {} 
\sphinxAtStartPar
\sphinxcode{\sphinxupquote{db: Session}}: Сессия базы данных, полученная из контекста зависимости \sphinxcode{\sphinxupquote{get\_db}}.

\end{itemize}

\sphinxAtStartPar
Действия:
\begin{enumerate}
\sphinxsetlistlabels{\arabic}{enumi}{enumii}{}{.}%
\item {} 
\sphinxAtStartPar
Вызывает функцию \sphinxcode{\sphinxupquote{db\_manager.add\_user\_dog\_collar}} для добавления новой связи в базу данных.

\item {} 
\sphinxAtStartPar
Если функция \sphinxcode{\sphinxupquote{db\_manager.add\_user\_dog\_collar}} успешно добавляет связь, функция записывает информацию об успешном добавлении в лог и возвращает ID пользователя, ID ошейника и сообщение об успешном добавлении.

\item {} 
\sphinxAtStartPar
Обрабатывает любые исключения, возникшие в процессе добавления, записывает их в лог и генерирует исключение \sphinxcode{\sphinxupquote{HTTPException}} с кодом 500 и описанием ошибки.

\end{enumerate}

\sphinxAtStartPar
\sphinxstylestrong{10. \sphinxcode{\sphinxupquote{update\_user\_dog\_collar(binding\_id: int, binding\_update: UserDogCollarUpdate, db: Session = Depends(get\_db))}}}

\sphinxAtStartPar
Описание: Эта функция отвечает за обновление данных существующей связи между пользователем и ошейником.

\sphinxAtStartPar
Параметры:
\begin{itemize}
\item {} 
\sphinxAtStartPar
\sphinxcode{\sphinxupquote{binding\_id: int}}: ID связи, которую нужно обновить.

\item {} 
\sphinxAtStartPar
\sphinxcode{\sphinxupquote{binding\_update: UserDogCollarUpdate}}: Данные для обновления, которые передаются в виде объекта \sphinxcode{\sphinxupquote{UserDogCollarUpdate}}.

\item {} 
\sphinxAtStartPar
\sphinxcode{\sphinxupquote{db: Session}}: Сессия базы данных, полученная из контекста зависимости \sphinxcode{\sphinxupquote{get\_db}}.

\end{itemize}

\sphinxAtStartPar
Действия:
\begin{enumerate}
\sphinxsetlistlabels{\arabic}{enumi}{enumii}{}{.}%
\item {} 
\sphinxAtStartPar
Вызывает функцию \sphinxcode{\sphinxupquote{db\_manager.update\_user\_dog\_collar}} для обновления данных связи в базе данных.

\item {} 
\sphinxAtStartPar
Если функция \sphinxcode{\sphinxupquote{db\_manager.update\_user\_dog\_collar}} успешно обновляет данные, функция записывает информацию об успешном обновлении в лог и возвращает сообщение об успешном обновлении.

\item {} 
\sphinxAtStartPar
Обрабатывает любые исключения, возникшие в процессе обновления, записывает их в лог и генерирует исключение \sphinxcode{\sphinxupquote{HTTPException}} с кодом 500 и описанием ошибки.

\end{enumerate}

\sphinxAtStartPar
\sphinxstylestrong{11. \sphinxcode{\sphinxupquote{delete\_user\_dog\_collar(binding\_id: int, db: Session = Depends(get\_db))}}}

\sphinxAtStartPar
Описание: Эта функция отвечает за удаление связи между пользователем и ошейником.

\sphinxAtStartPar
Параметры:
\begin{itemize}
\item {} 
\sphinxAtStartPar
\sphinxcode{\sphinxupquote{binding\_id: int}}: ID связи, которую нужно удалить.

\item {} 
\sphinxAtStartPar
\sphinxcode{\sphinxupquote{db: Session}}: Сессия базы данных, полученная из контекста зависимости \sphinxcode{\sphinxupquote{get\_db}}.

\end{itemize}

\sphinxAtStartPar
Действия:
\begin{enumerate}
\sphinxsetlistlabels{\arabic}{enumi}{enumii}{}{.}%
\item {} 
\sphinxAtStartPar
Вызывает функцию \sphinxcode{\sphinxupquote{db\_manager.archive\_user\_dog\_collar}} для удаления связи из базы данных.

\item {} 
\sphinxAtStartPar
Если функция \sphinxcode{\sphinxupquote{db\_manager.archive\_user\_dog\_collar}} успешно удаляет связь, функция записывает информацию об успешном удалении в лог и возвращает сообщение об успешном удалении.

\item {} 
\sphinxAtStartPar
Обрабатывает любые исключения, возникшие в процессе удаления, записывает их в лог и генерирует исключение \sphinxcode{\sphinxupquote{HTTPException}} с кодом 500 и описанием ошибки.

\end{enumerate}


\section{Дополнительные замечания:}
\label{\detokenize{src/devices/devices:id2}}\begin{itemize}
\item {} 
\sphinxAtStartPar
\sphinxcode{\sphinxupquote{db\_manager}}: Класс, который предоставляет функции для взаимодействия с базой данных.

\item {} 
\sphinxAtStartPar
\sphinxcode{\sphinxupquote{log\_message}}:  Функция для записи сообщений в лог\sphinxhyphen{}файл.

\item {} 
\sphinxAtStartPar
\sphinxcode{\sphinxupquote{HTTPException}}:  Это исключение используется для генерации ответа HTTP с кодом ошибки.

\end{itemize}


\section{Пример использования:}
\label{\detokenize{src/devices/devices:id3}}
\begin{sphinxVerbatim}[commandchars=\\\{\}]
\PYG{c+c1}{\PYGZsh{} Создание нового ошейника}
\PYG{n}{collar\PYGZus{}data} \PYG{o}{=} \PYG{n}{DogCollarCreate}\PYG{p}{(}\PYG{n}{uni\PYGZus{}num\PYGZus{}dog}\PYG{o}{=}\PYG{l+s+s2}{\PYGZdq{}}\PYG{l+s+s2}{1234567890}\PYG{l+s+s2}{\PYGZdq{}}\PYG{p}{,} \PYG{n}{name\PYGZus{}dog}\PYG{o}{=}\PYG{l+s+s2}{\PYGZdq{}}\PYG{l+s+s2}{Rex}\PYG{l+s+s2}{\PYGZdq{}}\PYG{p}{,} \PYG{o}{.}\PYG{o}{.}\PYG{o}{.}\PYG{p}{)}
\PYG{n}{collar\PYGZus{}id}\PYG{p}{,} \PYG{n}{message} \PYG{o}{=} \PYG{n}{create\PYGZus{}dog\PYGZus{}collar}\PYG{p}{(}\PYG{n}{collar\PYGZus{}data}\PYG{p}{,} \PYG{n}{db}\PYG{p}{)}

\PYG{c+c1}{\PYGZsh{} Обновление данных ошейника}
\PYG{n}{collar\PYGZus{}update\PYGZus{}data} \PYG{o}{=} \PYG{n}{DogCollarUpdate}\PYG{p}{(}\PYG{n}{name\PYGZus{}dog}\PYG{o}{=}\PYG{l+s+s2}{\PYGZdq{}}\PYG{l+s+s2}{Rexy}\PYG{l+s+s2}{\PYGZdq{}}\PYG{p}{,} \PYG{o}{.}\PYG{o}{.}\PYG{o}{.}\PYG{p}{)}
\PYG{n}{update\PYGZus{}dog\PYGZus{}collar}\PYG{p}{(}\PYG{n}{collar\PYGZus{}id}\PYG{p}{,} \PYG{n}{collar\PYGZus{}update\PYGZus{}data}\PYG{p}{,} \PYG{n}{db}\PYG{p}{)}

\PYG{c+c1}{\PYGZsh{} Удаление (архивирование) ошейника}
\PYG{n}{delete\PYGZus{}dog\PYGZus{}collar}\PYG{p}{(}\PYG{n}{collar\PYGZus{}id}\PYG{p}{,} \PYG{n}{db}\PYG{p}{)}

\PYG{c+c1}{\PYGZsh{} Получение списка всех ошейников}
\PYG{n}{all\PYGZus{}collars} \PYG{o}{=} \PYG{n}{get\PYGZus{}all\PYGZus{}dog\PYGZus{}collars}\PYG{p}{(}\PYG{n}{db}\PYG{p}{)}

\PYG{c+c1}{\PYGZsh{} Создание новых координат}
\PYG{n}{coordinate\PYGZus{}data} \PYG{o}{=} \PYG{n}{CoordinateCreate}\PYG{p}{(}\PYG{n}{collar\PYGZus{}id}\PYG{o}{=}\PYG{n}{collar\PYGZus{}id}\PYG{p}{,} \PYG{n}{latitude}\PYG{o}{=}\PYG{l+m+mf}{55.7558}\PYG{p}{,} \PYG{n}{longitude}\PYG{o}{=}\PYG{l+m+mf}{37.6173}\PYG{p}{)}
\PYG{n}{coordinate\PYGZus{}id}\PYG{p}{,} \PYG{n}{message} \PYG{o}{=} \PYG{n}{create\PYGZus{}coordinate}\PYG{p}{(}\PYG{n}{coordinate\PYGZus{}data}\PYG{p}{,} \PYG{n}{db}\PYG{p}{)}

\PYG{c+c1}{\PYGZsh{} Обновление данных координат}
\PYG{n}{coordinate\PYGZus{}update\PYGZus{}data} \PYG{o}{=} \PYG{n}{CoordinateUpdate}\PYG{p}{(}\PYG{n}{latitude}\PYG{o}{=}\PYG{l+m+mf}{55.7560}\PYG{p}{,} \PYG{n}{longitude}\PYG{o}{=}\PYG{l+m+mf}{37.6175}\PYG{p}{)}
\PYG{n}{update\PYGZus{}coordinate}\PYG{p}{(}\PYG{n}{coordinate\PYGZus{}id}\PYG{p}{,} \PYG{n}{coordinate\PYGZus{}update\PYGZus{}data}\PYG{p}{,} \PYG{n}{db}\PYG{p}{)}

\PYG{c+c1}{\PYGZsh{} Удаление координат}
\PYG{n}{delete\PYGZus{}coordinate}\PYG{p}{(}\PYG{n}{coordinate\PYGZus{}id}\PYG{p}{,} \PYG{n}{db}\PYG{p}{)}

\PYG{c+c1}{\PYGZsh{} Получение списка координат для ошейника за определенный период}
\PYG{n}{track} \PYG{o}{=} \PYG{n}{get\PYGZus{}track}\PYG{p}{(}\PYG{n}{collar\PYGZus{}id}\PYG{p}{,} \PYG{l+s+s2}{\PYGZdq{}}\PYG{l+s+s2}{2023\PYGZhy{}10\PYGZhy{}27T10:00:00Z}\PYG{l+s+s2}{\PYGZdq{}}\PYG{p}{,} \PYG{l+s+s2}{\PYGZdq{}}\PYG{l+s+s2}{2023\PYGZhy{}10\PYGZhy{}27T12:00:00Z}\PYG{l+s+s2}{\PYGZdq{}}\PYG{p}{,} \PYG{n}{db}\PYG{p}{)}

\PYG{c+c1}{\PYGZsh{} Создание связи между пользователем и ошейником}
\PYG{n}{binding\PYGZus{}data} \PYG{o}{=} \PYG{n}{UserDogCollarCreate}\PYG{p}{(}\PYG{n}{id\PYGZus{}user}\PYG{o}{=}\PYG{l+m+mi}{123}\PYG{p}{,} \PYG{n}{id\PYGZus{}collar}\PYG{o}{=}\PYG{n}{collar\PYGZus{}id}\PYG{p}{)}
\PYG{n}{userid}\PYG{p}{,}\PYG{n}{dogid}\PYG{p}{,}\PYG{n}{message} \PYG{o}{=} \PYG{n}{create\PYGZus{}user\PYGZus{}dog\PYGZus{}collar}\PYG{p}{(}\PYG{n}{binding\PYGZus{}data}\PYG{p}{,} \PYG{n}{db}\PYG{p}{)}

\PYG{c+c1}{\PYGZsh{} Обновление данных связи}
\PYG{n}{binding\PYGZus{}update\PYGZus{}data} \PYG{o}{=} \PYG{n}{UserDogCollarUpdate}\PYG{p}{(}\PYG{n}{id\PYGZus{}user}\PYG{o}{=}\PYG{l+m+mi}{123}\PYG{p}{,} \PYG{n}{id\PYGZus{}collar}\PYG{o}{=}\PYG{l+m+mi}{901}\PYG{p}{)}
\PYG{n}{update\PYGZus{}user\PYGZus{}dog\PYGZus{}collar}\PYG{p}{(}\PYG{n}{binding\PYGZus{}id}\PYG{p}{,} \PYG{n}{binding\PYGZus{}update\PYGZus{}data}\PYG{p}{,} \PYG{n}{db}\PYG{p}{)}

\PYG{c+c1}{\PYGZsh{} Удаление связи}
\PYG{n}{delete\PYGZus{}user\PYGZus{}dog\PYGZus{}collar}\PYG{p}{(}\PYG{n}{binding\PYGZus{}id}\PYG{p}{,} \PYG{n}{db}\PYG{p}{)}
\end{sphinxVerbatim}

\sphinxstepscope


\chapter{Функции для работы с заданиями}
\label{\detokenize{src/tasks/tasks:id1}}\label{\detokenize{src/tasks/tasks::doc}}\begin{enumerate}
\sphinxsetlistlabels{\arabic}{enumi}{enumii}{}{.}%
\item {} 
\sphinxAtStartPar
\sphinxcode{\sphinxupquote{create\_task(task: TaskCreate, db: Session = Depends(get\_db))}}

\end{enumerate}

\sphinxAtStartPar
Описание: Эта функция отвечает за создание новых заданий.

\sphinxAtStartPar
Параметры:
\begin{itemize}
\item {} 
\sphinxAtStartPar
\sphinxcode{\sphinxupquote{task: TaskCreate}}: Данные нового задания, которые передаются в виде объекта \sphinxcode{\sphinxupquote{TaskCreate}}.

\item {} 
\sphinxAtStartPar
\sphinxcode{\sphinxupquote{db: Session}}: Сессия базы данных, полученная из контекста зависимости \sphinxcode{\sphinxupquote{get\_db}}.

\end{itemize}

\sphinxAtStartPar
Действия:
\begin{enumerate}
\sphinxsetlistlabels{\arabic}{enumi}{enumii}{}{.}%
\item {} 
\sphinxAtStartPar
Вызывает функцию \sphinxcode{\sphinxupquote{db\_manager.add\_task}} для добавления нового задания в базу данных.

\item {} 
\sphinxAtStartPar
Если функция \sphinxcode{\sphinxupquote{db\_manager.add\_task}} успешно добавляет задание, функция записывает информацию об успешном добавлении в лог и возвращает ID задания, сообщение об успешном добавлении и текст задания.

\item {} 
\sphinxAtStartPar
Обрабатывает любые исключения, возникшие в процессе добавления, записывает их в лог и генерирует исключение \sphinxcode{\sphinxupquote{HTTPException}} с кодом 500 и описанием ошибки.

\item {} 
\sphinxAtStartPar
\sphinxcode{\sphinxupquote{update\_task(task\_id: int, task\_update: TaskUpdate, db: Session = Depends(get\_db))}}

\end{enumerate}

\sphinxAtStartPar
Описание: Эта функция отвечает за обновление данных существующего задания.

\sphinxAtStartPar
Параметры:
\begin{itemize}
\item {} 
\sphinxAtStartPar
\sphinxcode{\sphinxupquote{task\_id: int}}: ID задания, которое нужно обновить.

\item {} 
\sphinxAtStartPar
\sphinxcode{\sphinxupquote{task\_update: TaskUpdate}}: Данные для обновления, которые передаются в виде объекта \sphinxcode{\sphinxupquote{TaskUpdate}}.

\item {} 
\sphinxAtStartPar
\sphinxcode{\sphinxupquote{db: Session}}: Сессия базы данных, полученная из контекста зависимости \sphinxcode{\sphinxupquote{get\_db}}.

\end{itemize}

\sphinxAtStartPar
Действия:
\begin{enumerate}
\sphinxsetlistlabels{\arabic}{enumi}{enumii}{}{.}%
\item {} 
\sphinxAtStartPar
Вызывает функцию \sphinxcode{\sphinxupquote{db\_manager.update\_task}} для обновления данных задания в базе данных.

\item {} 
\sphinxAtStartPar
Если функция \sphinxcode{\sphinxupquote{db\_manager.update\_task}} успешно обновляет данные, функция записывает информацию об успешном обновлении в лог и возвращает сообщение об успешном обновлении.

\item {} 
\sphinxAtStartPar
Обрабатывает любые исключения, возникшие в процессе обновления, записывает их в лог и генерирует исключение \sphinxcode{\sphinxupquote{HTTPException}} с кодом 500 и описанием ошибки.

\item {} 
\sphinxAtStartPar
\sphinxcode{\sphinxupquote{delete\_task(task\_id: int, db: Session = Depends(get\_db))}}

\end{enumerate}

\sphinxAtStartPar
Описание: Эта функция отвечает за удаление (архивацию) задания.

\sphinxAtStartPar
Параметры:
\begin{itemize}
\item {} 
\sphinxAtStartPar
\sphinxcode{\sphinxupquote{task\_id: int}}: ID задания, которое нужно удалить.

\item {} 
\sphinxAtStartPar
\sphinxcode{\sphinxupquote{db: Session}}: Сессия базы данных, полученная из контекста зависимости \sphinxcode{\sphinxupquote{get\_db}}.

\end{itemize}

\sphinxAtStartPar
Действия:
\begin{enumerate}
\sphinxsetlistlabels{\arabic}{enumi}{enumii}{}{.}%
\item {} 
\sphinxAtStartPar
Вызывает функцию \sphinxcode{\sphinxupquote{db\_manager.archive\_task}} для архивирования задания в базе данных.

\item {} 
\sphinxAtStartPar
Если функция \sphinxcode{\sphinxupquote{db\_manager.archive\_task}} успешно архивирует задание, функция записывает информацию об успешном удалении в лог и возвращает сообщение об успешном удалении.

\item {} 
\sphinxAtStartPar
Обрабатывает любые исключения, возникшие в процессе удаления, записывает их в лог и генерирует исключение \sphinxcode{\sphinxupquote{HTTPException}} с кодом 500 и описанием ошибки.

\end{enumerate}


\section{Дополнительные замечания:}
\label{\detokenize{src/tasks/tasks:id2}}\begin{itemize}
\item {} 
\sphinxAtStartPar
\sphinxcode{\sphinxupquote{db\_manager}}: Класс, который предоставляет функции для взаимодействия с базой данных.

\item {} 
\sphinxAtStartPar
\sphinxcode{\sphinxupquote{log\_message}}:  Функция для записи сообщений в лог\sphinxhyphen{}файл.

\item {} 
\sphinxAtStartPar
\sphinxcode{\sphinxupquote{HTTPException}}:  Это исключение используется для генерации ответа HTTP с кодом ошибки.

\end{itemize}


\section{Пример использования:}
\label{\detokenize{src/tasks/tasks:id3}}
\begin{sphinxVerbatim}[commandchars=\\\{\}]
\PYG{c+c1}{\PYGZsh{} Создание нового задания}
\PYG{n}{task\PYGZus{}data} \PYG{o}{=} \PYG{n}{TaskCreate}\PYG{p}{(}\PYG{n}{id\PYGZus{}user\PYGZus{}1}\PYG{o}{=}\PYG{l+m+mi}{123}\PYG{p}{,} \PYG{n}{id\PYGZus{}user\PYGZus{}2}\PYG{o}{=}\PYG{l+m+mi}{456}\PYG{p}{,} \PYG{n}{id\PYGZus{}task}\PYG{o}{=}\PYG{l+m+mi}{789}\PYG{p}{,} \PYG{n}{task\PYGZus{}text}\PYG{o}{=}\PYG{l+s+s2}{\PYGZdq{}}\PYG{l+s+s2}{Купить хлеб}\PYG{l+s+s2}{\PYGZdq{}}\PYG{p}{)}
\PYG{n}{task\PYGZus{}id}\PYG{p}{,} \PYG{n}{message}\PYG{p}{,} \PYG{n}{task\PYGZus{}text} \PYG{o}{=} \PYG{n}{create\PYGZus{}task}\PYG{p}{(}\PYG{n}{task\PYGZus{}data}\PYG{p}{,} \PYG{n}{db}\PYG{p}{)}

\PYG{c+c1}{\PYGZsh{} Обновление данных задания}
\PYG{n}{task\PYGZus{}update\PYGZus{}data} \PYG{o}{=} \PYG{n}{TaskUpdate}\PYG{p}{(}\PYG{n}{confirm}\PYG{o}{=}\PYG{k+kc}{True}\PYG{p}{,} \PYG{n}{task\PYGZus{}text}\PYG{o}{=}\PYG{l+s+s2}{\PYGZdq{}}\PYG{l+s+s2}{Купить молоко}\PYG{l+s+s2}{\PYGZdq{}}\PYG{p}{)}
\PYG{n}{update\PYGZus{}task}\PYG{p}{(}\PYG{n}{task\PYGZus{}id}\PYG{p}{,} \PYG{n}{task\PYGZus{}update\PYGZus{}data}\PYG{p}{,} \PYG{n}{db}\PYG{p}{)}

\PYG{c+c1}{\PYGZsh{} Удаление (архивирование) задания}
\PYG{n}{delete\PYGZus{}task}\PYG{p}{(}\PYG{n}{task\PYGZus{}id}\PYG{p}{,} \PYG{n}{db}\PYG{p}{)}
\end{sphinxVerbatim}

\sphinxstepscope


\chapter{Функции для работы с пользователями}
\label{\detokenize{src/users/users:id1}}\label{\detokenize{src/users/users::doc}}\begin{enumerate}
\sphinxsetlistlabels{\arabic}{enumi}{enumii}{}{.}%
\item {} 
\sphinxAtStartPar
\sphinxcode{\sphinxupquote{create\_user(user: UserCreate, db: Session = Depends(get\_db))}}

\end{enumerate}

\sphinxAtStartPar
Описание: Эта функция отвечает за регистрацию новых пользователей.

\sphinxAtStartPar
Параметры:
\begin{itemize}
\item {} 
\sphinxAtStartPar
\sphinxcode{\sphinxupquote{user: UserCreate}}: Данные нового пользователя, которые передаются в виде объекта \sphinxcode{\sphinxupquote{UserCreate}}.

\item {} 
\sphinxAtStartPar
\sphinxcode{\sphinxupquote{db: Session}}: Сессия базы данных, полученная из контекста зависимости \sphinxcode{\sphinxupquote{get\_db}}.

\end{itemize}

\sphinxAtStartPar
Действия:
\begin{enumerate}
\sphinxsetlistlabels{\arabic}{enumi}{enumii}{}{.}%
\item {} 
\sphinxAtStartPar
Хеширует пароль пользователя с помощью функции \sphinxcode{\sphinxupquote{get\_password\_hash}}.

\item {} 
\sphinxAtStartPar
Создает словарь \sphinxcode{\sphinxupquote{user\_data}} из объекта \sphinxcode{\sphinxupquote{user}}, исключая поле \sphinxcode{\sphinxupquote{deletion\_date}}.

\item {} 
\sphinxAtStartPar
Заменяет поле \sphinxcode{\sphinxupquote{password}} в словаре \sphinxcode{\sphinxupquote{user\_data}} на хешированный пароль.

\item {} 
\sphinxAtStartPar
Вызывает функцию \sphinxcode{\sphinxupquote{db\_manager.add\_user}} для добавления нового пользователя в базу данных.

\item {} 
\sphinxAtStartPar
Если функция \sphinxcode{\sphinxupquote{db\_manager.add\_user}} возвращает \sphinxcode{\sphinxupquote{user\_id}} равный 0, то это означает ошибку. В этом случае функция записывает ошибку в лог с помощью функции \sphinxcode{\sphinxupquote{log\_message}} и генерирует исключение \sphinxcode{\sphinxupquote{HTTPException}} с кодом 500 и описанием ошибки.

\item {} 
\sphinxAtStartPar
Если функция \sphinxcode{\sphinxupquote{db\_manager.add\_user}} успешно добавляет пользователя, функция записывает информацию об успешной регистрации в лог и возвращает \sphinxcode{\sphinxupquote{user\_id}} и сообщение об успешной регистрации.

\item {} 
\sphinxAtStartPar
Обрабатывает любые исключения, возникшие в процессе регистрации, записывает их в лог и генерирует исключение \sphinxcode{\sphinxupquote{HTTPException}} с кодом 500 и описанием ошибки.

\item {} 
\sphinxAtStartPar
\sphinxcode{\sphinxupquote{update\_user(user\_id: int, user\_update: UserUpdate, db: Session = Depends(get\_db))}}

\end{enumerate}

\sphinxAtStartPar
Описание: Эта функция отвечает за обновление данных существующего пользователя.

\sphinxAtStartPar
Параметры:
\begin{itemize}
\item {} 
\sphinxAtStartPar
\sphinxcode{\sphinxupquote{user\_id: int}}: ID пользователя, которого нужно обновить.

\item {} 
\sphinxAtStartPar
\sphinxcode{\sphinxupquote{user\_update: UserUpdate}}: Данные для обновления, которые передаются в виде объекта \sphinxcode{\sphinxupquote{UserUpdate}}.

\item {} 
\sphinxAtStartPar
\sphinxcode{\sphinxupquote{db: Session}}: Сессия базы данных, полученная из контекста зависимости \sphinxcode{\sphinxupquote{get\_db}}.

\end{itemize}

\sphinxAtStartPar
Действия:
\begin{enumerate}
\sphinxsetlistlabels{\arabic}{enumi}{enumii}{}{.}%
\item {} 
\sphinxAtStartPar
Вызывает функцию \sphinxcode{\sphinxupquote{db\_manager.update\_user}} для обновления данных пользователя в базе данных.

\item {} 
\sphinxAtStartPar
Если функция \sphinxcode{\sphinxupquote{db\_manager.update\_user}} успешно обновляет данные, функция записывает информацию об успешном обновлении в лог и возвращает сообщение об успешном обновлении.

\item {} 
\sphinxAtStartPar
Обрабатывает любые исключения, возникшие в процессе обновления, записывает их в лог и генерирует исключение \sphinxcode{\sphinxupquote{HTTPException}} с кодом 500 и описанием ошибки.

\item {} 
\sphinxAtStartPar
\sphinxcode{\sphinxupquote{delete\_user(user\_id: int, db: Session = Depends(get\_db))}}

\end{enumerate}

\sphinxAtStartPar
Описание: Эта функция отвечает за удаление (архивацию) пользователя.

\sphinxAtStartPar
Параметры:
\begin{itemize}
\item {} 
\sphinxAtStartPar
\sphinxcode{\sphinxupquote{user\_id: int}}: ID пользователя, которого нужно удалить.

\item {} 
\sphinxAtStartPar
\sphinxcode{\sphinxupquote{db: Session}}: Сессия базы данных, полученная из контекста зависимости \sphinxcode{\sphinxupquote{get\_db}}.

\end{itemize}

\sphinxAtStartPar
Действия:
\begin{enumerate}
\sphinxsetlistlabels{\arabic}{enumi}{enumii}{}{.}%
\item {} 
\sphinxAtStartPar
Вызывает функцию \sphinxcode{\sphinxupquote{db\_manager.archive\_user}} для архивирования пользователя в базе данных.

\item {} 
\sphinxAtStartPar
Если функция \sphinxcode{\sphinxupquote{db\_manager.archive\_user}} успешно архивирует пользователя, функция записывает информацию об успешном удалении в лог и возвращает сообщение об успешном удалении.

\item {} 
\sphinxAtStartPar
Обрабатывает любые исключения, возникшие в процессе удаления, записывает их в лог и генерирует исключение \sphinxcode{\sphinxupquote{HTTPException}} с кодом 500 и описанием ошибки.

\end{enumerate}


\section{Дополнительные замечания:}
\label{\detokenize{src/users/users:id2}}\begin{itemize}
\item {} 
\sphinxAtStartPar
\sphinxcode{\sphinxupquote{db\_manager}}: Класс, который предоставляет функции для взаимодействия с базой данных.

\item {} 
\sphinxAtStartPar
\sphinxcode{\sphinxupquote{log\_message}}:  Функция для записи сообщений в лог\sphinxhyphen{}файл.

\item {} 
\sphinxAtStartPar
\sphinxcode{\sphinxupquote{get\_password\_hash}}:  Функция для хеширования паролей.

\item {} 
\sphinxAtStartPar
\sphinxcode{\sphinxupquote{HTTPException}}:  Это исключение используется для генерации ответа HTTP с кодом ошибки.

\end{itemize}


\section{Пример использования:}
\label{\detokenize{src/users/users:id3}}
\begin{sphinxVerbatim}[commandchars=\\\{\}]
\PYG{c+c1}{\PYGZsh{} Регистрация нового пользователя}
\PYG{n}{user\PYGZus{}data} \PYG{o}{=} \PYG{n}{UserCreate}\PYG{p}{(}\PYG{n}{login}\PYG{o}{=}\PYG{l+s+s2}{\PYGZdq{}}\PYG{l+s+s2}{JohnDoe}\PYG{l+s+s2}{\PYGZdq{}}\PYG{p}{,} \PYG{n}{password}\PYG{o}{=}\PYG{l+s+s2}{\PYGZdq{}}\PYG{l+s+s2}{password123}\PYG{l+s+s2}{\PYGZdq{}}\PYG{p}{,} \PYG{n}{email}\PYG{o}{=}\PYG{l+s+s2}{\PYGZdq{}}\PYG{l+s+s2}{johndoe@example.com}\PYG{l+s+s2}{\PYGZdq{}}\PYG{p}{,} \PYG{o}{.}\PYG{o}{.}\PYG{o}{.}\PYG{p}{)}
\PYG{n}{user\PYGZus{}id}\PYG{p}{,} \PYG{n}{message} \PYG{o}{=} \PYG{n}{create\PYGZus{}user}\PYG{p}{(}\PYG{n}{user\PYGZus{}data}\PYG{p}{,} \PYG{n}{db}\PYG{p}{)}

\PYG{c+c1}{\PYGZsh{} Обновление данных пользователя}
\PYG{n}{user\PYGZus{}update\PYGZus{}data} \PYG{o}{=} \PYG{n}{UserUpdate}\PYG{p}{(}\PYG{n}{phone}\PYG{o}{=}\PYG{l+s+s2}{\PYGZdq{}}\PYG{l+s+s2}{88001234567}\PYG{l+s+s2}{\PYGZdq{}}\PYG{p}{)}
\PYG{n}{update\PYGZus{}user}\PYG{p}{(}\PYG{n}{user\PYGZus{}id}\PYG{p}{,} \PYG{n}{user\PYGZus{}update\PYGZus{}data}\PYG{p}{,} \PYG{n}{db}\PYG{p}{)}

\PYG{c+c1}{\PYGZsh{} Удаление (архивирование) пользователя}
\PYG{n}{delete\PYGZus{}user}\PYG{p}{(}\PYG{n}{user\PYGZus{}id}\PYG{p}{,} \PYG{n}{db}\PYG{p}{)}
\end{sphinxVerbatim}



\renewcommand{\indexname}{Алфавитный указатель}
\printindex
\end{document}